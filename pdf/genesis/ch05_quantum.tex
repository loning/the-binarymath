
\section{The Necessary Emergence of Observers from Self-Referential Completeness}
\label{sec:ch05_quantum:the-necessary-emergence-of-observers-from-self-referential-completeness}

\textbf{Theoretical Viewpoint}: In our framework, observers appear as intrinsic structures of self-referential completeness.

\textbf{Theorem~\ref{thm:3.1} (Observer Emergence Theorem)}
\label{thm:3.1}
Self-referentially complete systems necessarily generate internal observer structures.

\textbf{Proof}:

\begin{enumerate}
\item \textbf{Dynamic Requirements of Self-Reference}:
\end{enumerate}
   From dynamic self-referential completeness (Definition 1.2):
   
\begin{equation}
\text{DynamicSelfRef}(S) \equiv \forall t: \text{SelfRefComplete}(S_t) \land S_{t+1} = \Phi(S_t)
\end{equation}
   The execution of evolution operator $\Phi$ requires some "mechanism".
   
\begin{enumerate}
\item \textbf{Internality of the Execution Mechanism}:
\end{enumerate}
   Let $\mathcal{M}$ be the mechanism executing $\Phi$. By self-referential completeness:
   
\begin{equation}
\mathcal{M} \in S
\end{equation}
   (Otherwise the system's description would be incomplete)
   
\begin{enumerate}
\item \textbf{Functional Analysis of the Mechanism}:
\end{enumerate}
   $\mathcal{M}$ must be able to:
   - Read current state: $\text{read}: S_t \to \text{Info}(S_t)$
   - Compute new descriptions: $\text{compute}: \text{Info}(S_t) \to \text{Desc}^{(t+1)}$
   - Update the system: $\text{update}: S_t \times \text{Desc}^{(t+1)} \to S_{t+1}$
   
   \textbf{Key Clarification}: These three functions must be implemented simultaneously in $\mathcal{M}$ because:
   - Self-reference requires $\mathcal{M}$ to describe its own functions (recursivity)
   - Completeness requires $\mathcal{M}$ to handle all possible state transitions
   - Dynamics requires $\mathcal{M}$ to actually execute these operations at each moment
   
\begin{enumerate}
\item \textbf{Rigorous Derivation of the Observer Concept}:
\end{enumerate}
   \textbf{Lemma 5.1.1}: The mechanism $\mathcal{M}$ necessarily has observer characteristics
\label{thm:3.1}
   
   \textbf{Proof}:
   - The read function requires $\mathcal{M}$ to distinguish different states, i.e., "observation" capability
   - The compute function requires $\mathcal{M}$ to process acquired information, i.e., "cognition" capability
   - The update function requires $\mathcal{M}$ to affect system states, i.e., "action" capability
   
   \textbf{Key Insight}: observation, cognition, action = complete definition of observer
   
   Therefore, $\mathcal{M}$ is not just a mechanism, but an observer. $\square$

\begin{enumerate}
\item \textbf{Formal Definition of Observer}:
\end{enumerate}
   
\begin{equation}
O = \{o \in S: o \text{ can execute the } \text{read} \circ \text{compute} \circ \text{update} \text{ sequence}\}
\end{equation}

   \textbf{Lemma 5.2.2}: Existence and uniqueness of observer $O$
\label{thm:3.1}
   
   \textbf{Proof}:
   - Existence: From steps 1-4, $\mathcal{M} \in O$, so $O \neq \emptyset$
   - Uniqueness: If there were two different observers $o_1, o_2$, they must distinguish all states, but this would lead to non-injectivity of the description function, violating self-referential completeness
   - Therefore, at each moment $t$, there exists a unique observer $O_t$ $\square$

\begin{enumerate}
\item \textbf{Necessity of the Observer}:
\end{enumerate}
   Since the evolution operator $\Phi$ must execute at each moment, the observer $O$ must exist at each moment.
   
   Therefore, self-referentially complete systems necessarily generate internal observer structures. $\square$

\section{Mathematical Structure of Observers}
\label{sec:ch05_quantum:mathematical-structure-of-observers}

\textbf{Definition 5.1 (Formal Definition of Observer)}
\label{thm:3.1}
An observer is a subsystem in a self-referential system that performs measurement operations:
\begin{equation}
O = (S_O, \mathcal{A}_O, \mathcal{M}_O)
\end{equation}
where:
\begin{itemize}
\item $S_O \subseteq S$: State space occupied by the observer
\item $\mathcal{A}_O$: Action set of the observer (measurement choices)
\item $\mathcal{M}_O: S \times \mathcal{A}_O \to \mathcal{R}$: Measurement mapping to result space
\end{itemize}

\textbf{Theorem~\ref{thm:3.2} (Observer Paradox and Description Multiplicity)}
\label{thm:3.2}
An observer observing a system containing itself necessarily leads to description multiplicity.

\textbf{Proof}:

\begin{enumerate}
\item \textbf{Formalization of Self-Containment}:
\end{enumerate}
   Observer O observes system S, but $O \subseteq S$, therefore:
   
\begin{equation}
\text{Obs}(S) = \text{Obs}(S_{\text{other}} \cup O)
\end{equation}

\begin{enumerate}
\item \textbf{Recursive Expansion}:
\end{enumerate}
   A complete description needs to include the fact "O is observing S":
   
\begin{equation}
D_0 = \text{Desc}(S)
\end{equation}
   
\begin{equation}
D_1 = \text{Desc}(S) \cup \text{Desc}(\text{"O observing } S\text{"})
\end{equation}
   
\begin{equation}
D_2 = D_1 \cup \text{Desc}(\text{"O observing } D_1\text{"})
\end{equation}
   
\begin{equation}
\vdots
\end{equation}

\begin{enumerate}
\item \textbf{Necessity of Infinite Recursion}:
\end{enumerate}
   Each added layer of description changes the system state, requiring new description.
   Formally: $D_n \neq D_{n+1}$ for all $n \in \mathbb{N}$.
   
\begin{enumerate}
\item \textbf{Necessity of Finite Truncation}:
\end{enumerate}
   Actual observation must truncate at some layer $k$:
   
\begin{equation}
\hat{D}_k = D_k \text{ (ignoring higher layers)}
\end{equation}

\begin{enumerate}
\item \textbf{Emergence of Multiplicity}:
\end{enumerate}
   Different truncation choices $k$ give different descriptions:
   
\begin{equation}
\mathcal{D} = \{\hat{D}_0, \hat{D}_1, \hat{D}_2, ...\}
\end{equation}

Therefore, self-contained observation necessarily produces description multiplicity. $\square$

\section{The Necessary Emergence of Quantum Phenomena}
\label{sec:ch05_quantum:the-necessary-emergence-of-quantum-phenomena}

\textbf{Theorem~\ref{thm:3.3} (Necessity of Quantum Superposition)}
\label{thm:3.3}
Self-referentially complete systems necessarily exhibit quantum superposition states.

\textbf{Rigorous Proof}:

\begin{enumerate}
\item \textbf{From Description Multiplicity to Superposition}:
\end{enumerate}
   From Theorem~\ref{thm:3.2}, observation leads to description set $\mathcal{D} = {\hat{D}_0, \hat{D}_1, ...}$.
   
   \textbf{Lemma 5.3.1}: The state representation of an unobserved system necessarily takes a linear combination form.
\label{thm:3.3}
   
   \textbf{Proof}: Let the system's state before observation be $\psi$. We use Dirac notation $|\psi\rangle$ to represent the state vector, where $|\psi\rangle$ is the vector representation of $\psi$ in the description space $\mathcal{D}$.
   
   \textbf{Case Analysis}:
   - If $|\psi\rangle = |\hat{D}_k\rangle$ for some definite $k$, then the state is already determined, requiring no observation process, violating the necessity of observers (Theorem~\ref{thm:3.1})
   - If $|\psi\rangle$ is not equal to any $|\hat{D}_k\rangle$, then observation cannot produce any $\hat{D}_k$ state, violating the functionality of observers
   - If $|\psi\rangle$ equals multiple different $|\hat{D}_k\rangle$ simultaneously, this violates state uniqueness (self-referential completeness requires uniqueness of state descriptions)
   
   \textbf{Therefore, the only logical possibility}: $|\psi\rangle$ must be a linear combination of all possible description states:
   
\begin{equation}
|\psi\rangle = \sum_{k=0}^{\infty} \alpha_k |\hat{D}_k\rangle
\end{equation}
   
   where $\alpha_k$ are complex coefficients, and not all $\alpha_k$ are zero. $\square$

\begin{enumerate}
\item \textbf{Normalization Requirement}:
\end{enumerate}
   Since the system must be in some description state:
   
\begin{equation}
\sum_{k=0}^{\infty} |\alpha_k|^2 = 1
\end{equation}

\begin{enumerate}
\item \textbf{Necessity of Weight Coefficients}:
\end{enumerate}
   \textbf{Lemma 5.4.2}: Physical meaning of weight coefficients $\alpha_k$.
\label{thm:3.3}
   
   \textbf{Proof}: $|\alpha_k|^2$ represents the probability of the system collapsing to state $|\hat{D}_k\rangle$ after observation. This probability:
   - Must be non-negative: $|\alpha_k|^2 \geq 0$
   - Must be normalized: $\sum_k |\alpha_k|^2 = 1$
   - Reflects the "weight" or "accessibility" of each description level in the system
   
   The complex nature of coefficients $\alpha_k$ reflects possible phase relationships between different description levels, which is the mathematical embodiment of recursive structure.

Therefore, superposition is the mathematical expression of description multiplicity. $\square$

\textbf{Theorem~\ref{thm:3.4} (Necessity of Measurement Collapse)}
\label{thm:3.4}
Observer measurement necessarily leads to superposition state collapse.

\textbf{Proof}:

\begin{enumerate}
\item \textbf{Definition of Measurement}:
\end{enumerate}
   Observer O performing measurement means selecting a specific description $\hat{D}_m$ from $\mathcal{D}$.
   
\begin{enumerate}
\item \textbf{Exclusivity of Selection}:
\end{enumerate}
   Once $\hat{D}_m$ is selected, other descriptions are excluded:
   
\begin{equation}
|\psi\rangle \xrightarrow{\text{measurement}} |\hat{D}_m\rangle
\end{equation}

\begin{enumerate}
\item \textbf{Source of Selection Weights}:
\end{enumerate}
   Which $m$ is selected is influenced by the magnitude of coefficient $\alpha_m$.
   $|\alpha_m|^2$ reflects the weight of that description level in the system.
   
\begin{enumerate}
\item \textbf{Entropy-Based Irreversibility}:
\end{enumerate}
   \textbf{Lemma 5.5.1}: Irreversibility of measurement
\label{thm:3.4}
   
   \textbf{Proof}:
   - Before measurement: System state is superposition, containing multiple possibilities
   - After measurement: System state is definite, but includes "measurement record"
   - Record contains: selection result, selection time, observer state
   - Record is new information, increasing system entropy
   - By irreversibility of entropy increase, measurement is irreversible $\square$

Therefore, quantum collapse is the formal description of the observer's selection mechanism, and is a logical necessity of self-referentially complete systems. $\square$

\section{From Observers to Selection Weights}
\label{sec:ch05_quantum:from-observers-to-selection-weights}

\textbf{Theorem~\ref{thm:3.5} (Necessary Emergence of Selection Weights)}
\label{thm:3.5}
Observers in self-referential systems, when facing multiple descriptions, necessarily produce a selection weight distribution.

\textbf{Proof}:
Starting from the multiple descriptions arising from the observer paradox, we rigorously derive the necessity of weights.

\begin{enumerate}
\item \textbf{Precise Characterization of Multi-Value Situation}:
\end{enumerate}
   From Theorem~\ref{thm:3.2}, observation leads to description set $\mathcal{D} = {\hat{D}_0, \hat{D}_1, \hat{D}_2, ...}$
   
   Each $\hat{D}_k$ represents a description truncated at level $k$, containing self-referential information of different depths.

\begin{enumerate}
\item \textbf{Inevitability of Selection}:
\end{enumerate}
   
   \textbf{Lemma 5.6.1}: The observer must select a specific description in finite time.
\label{thm:3.5}
   
   \textbf{Proof} (based on entropy increase axiom and dynamic self-referential completeness):
   - Observation is a physical process requiring time
   - Infinite waiting means never completing observation
   - Self-referential systems require observation must complete (otherwise violating dynamic self-referential completeness $\text{DynamicSelfRef}(S)$)
   - Entropy increase axiom requires the system must continuously evolve: $\forall t: H(S_{t+1}) > H(S_t)$
   - A stalled observation process would prevent entropy increase, violating the fundamental axiom
   - Therefore must select $\hat{D}_k$ at some finite $k$

\begin{enumerate}
\item \textbf{Emergence Mechanism of Weights}:
\end{enumerate}
   
   The observer's selection probability is not arbitrary, but determined by the system's internal structure.
   
   \textbf{Lemma 5.7.2}: The constraint of completing observation in finite time leads to exponential weight distribution.
\label{thm:3.5}
   
   \textbf{Proof}: Rigorous derivation of cost function from self-referential completeness and entropy increase axiom.
   
   \textbf{Step 1: Theoretical Foundation of Recursive Cost}
   
   From Theorem~\ref{thm:3.2}, the observer must process description $\hat{D}_k$ with recursive depth $k$. Each recursive layer requires:
   - Executing description function $\text{Desc}$
   - Storing description results
   - Verifying self-referential consistency
   
   \textbf{Step 2: Deriving Computational Complexity from Self-Referential Completeness}
   
   From the definition of self-referential completeness, the description function $\text{Desc}: S \to \mathcal{L}$ must be injective.
   For layer $k$ description $\hat{D}_k$, the amount of information to process is $|D_k|$.
   
   \textbf{Key Observation}: Due to the injectivity of the description function, each added recursive layer requires processing at least one additional unit of information:
   
\begin{equation}
|D_k| \geq |D_{k-1}| + 1
\end{equation}
   Therefore, the minimum time complexity for computing layer $k$ description is:
   
\begin{equation}
T_k \geq \sum_{i=0}^{k-1} |D_i| \geq \sum_{i=0}^{k-1} i = \frac{k(k-1)}{2}
\end{equation}

   \textbf{Step 3: Deriving Storage Cost from Entropy Increase Axiom}
   
   The entropy increase axiom requires $H(S_{k+1}) > H(S_k)$, meaning each description layer contains more information.
   Under the most conservative estimate, each layer adds one unit of information:
   
\begin{equation}
H(S_k) \geq H(S_0) + k
\end{equation}
   Therefore storage cost is at least:
   
\begin{equation}
S_k \geq k
\end{equation}

   \textbf{Step 4: Lower Bound of Total Cost Function}
   
   Combining computation and storage costs, the total cost of layer $k$ description satisfies:
   
\begin{equation}
C_k \geq \alpha_1 \cdot \frac{k(k-1)}{2} + \alpha_2 \cdot k \geq \alpha k
\end{equation}
   where $\alpha$ is a positive constant determined by system parameters.
   
   \textbf{Maximum Entropy Principle}: As a logical consequence of the axiom and self-referential completeness definition, the system chooses the maximum entropy distribution:
   
\begin{equation}
\max_{\{w_k\}} \left\{ -\sum_k w_k \log w_k \right\} \quad \text{subject to } \sum_k w_k = 1, \sum_k w_k C_k = \bar{C}
\end{equation}
   
   Using Lagrange multipliers, we obtain:
   
\begin{equation}
w_k = \frac{e^{-\lambda C_k}}{Z} = \frac{e^{-\lambda \alpha k}}{Z}
\end{equation}
   
   where $Z = \sum_j e^{-\lambda \alpha j}$ is the partition function.
   
   Therefore the weight distribution is necessarily exponential:
   
\begin{equation}
w_k \propto \exp(-\alpha k)
\end{equation}
   
   After normalization, we obtain geometric distribution weights.

\begin{enumerate}
\item \textbf{Universal Form of Weight Distribution}:
\end{enumerate}
   
   More generally, considering the system's "selection potential function" $V_k$:
   
\begin{equation}
V_k = f(k, \text{system parameters})
\end{equation}
   
   where $f$ is determined by the system's specific structure. The weight distribution is:
   
\begin{equation}
w_k = \frac{\exp(-V_k)}{\sum_{j} \exp(-V_j)}
\end{equation}
   
   This form ensures normalization and positivity.

\begin{enumerate}
\item \textbf{Natural Satisfaction of Normalization}:
\end{enumerate}
   
   By definition, weights automatically satisfy:
\begin{equation}
\sum_{k=0}^{\infty} w_k = \sum_{k=0}^{\infty} \frac{\exp(-\beta E_k)}{\sum_{j} \exp(-\beta E_j)} = 1
\end{equation}

\begin{enumerate}
\item \textbf{Deep Meaning of Weight Distribution}:
\end{enumerate}
   
   The weight distribution reflects fundamental characteristics of self-referential systems:
   - \textbf{Finiteness constraint}: Infinite recursion must truncate in finite time
   - \textbf{Necessity of selection}: Observer must "decide" where to stop
   - \textbf{Origin of probability}: Uncertainty comes from choice of truncation point
   
   This explains the essence of quantum probability: it reflects intrinsic limitations of self-referential observation.

\begin{enumerate}
\item \textbf{Deep Structure of Recursion}:
\end{enumerate}
   
   More profoundly, weights themselves are self-referential:
   - Observer chooses description weights
   - This selection process itself needs description
   - Describing the selection process involves new weights
   - Forming weights of weights of weights...
   
   This infinite recursion is truncated by system finiteness, producing the quantum probabilities we observe.

Therefore, weight distribution is not an assumption, but a necessary emergence from self-referential structure. $\square$

\section{Weight Distribution and Collapse Mechanism}
\label{sec:ch05_quantum:weight-distribution-and-collapse-mechanism}

\textbf{Theorem~\ref{thm:3.6} (Equivalence of Weights and Collapse Probabilities)}
\label{thm:3.6}
The observer's selection weight distribution is mathematically equivalent to quantum measurement collapse probabilities.

\textbf{Rigorous Proof}:

\begin{enumerate}
\item \textbf{Structural Correspondence}:
\end{enumerate}
   - Observer selection: choosing one from description set ${\hat{D}_k}$
   - Quantum measurement: collapsing to one eigenstate from ${|\phi_k\rangle}$
   
\begin{enumerate}
\item \textbf{Satisfaction of Probability Axioms}:
\end{enumerate}
   Both satisfy Kolmogorov probability axioms:
   - Non-negativity: $w_k \geq 0$, $|\langle\phi_k|\psi\rangle|^2 \geq 0$
   - Normalization: $\sum_k w_k = 1$, $\sum_k |\langle\phi_k|\psi\rangle|^2 = 1$
   - Additivity: holds for disjoint events
   
\begin{enumerate}
\item \textbf{Dynamical Correspondence}:
\end{enumerate}
   - Selection weights determined by system structure: $w_k = f(\text{system parameters})$
   - Quantum probabilities determined by state vectors: $p_k = |\langle\phi_k|\psi\rangle|^2$
   
\begin{enumerate}
\item \textbf{Isomorphism}:
\end{enumerate}
   There exists a mapping $\Phi: {\hat{D}_k} \to {|\phi_k\rangle}$ such that:
   
\begin{equation}
w_k = |\langle\phi_k|\psi\rangle|^2
\end{equation}

Therefore, they are completely equivalent in mathematical structure. $\square$

\section{Derivation of Wave-Particle Duality}
\label{sec:ch05_quantum:derivation-of-wave-particle-duality}

\textbf{Theorem~\ref{thm:3.7} (Necessity of Wave-Particle Duality)}
\label{thm:3.7}
The type of observer in self-referential systems determines whether the system exhibits wave or particle properties.

\textbf{Rigorous Proof}:
Consider a system passing through two possible paths. Starting from self-referential completeness and observer necessity:

\textbf{Core Insight}: Wave-particle duality arises from different observer choices of description levels in self-referential systems.

\begin{enumerate}
\item \textbf{Description Levels of Path Information}:
\end{enumerate}
   
   According to Theorem~\ref{thm:3.2}, the system has multiple description layers ${\hat{D}_0, \hat{D}_1, \hat{D}_2, ...}$:
   - $\hat{D}_0$: System exists at some position (coarsest description)
   - $\hat{D}_1$: System reaches position through some paths (medium description)
   - $\hat{D}_2$: System reaches position through specific path $i$ (fine description)
   
\begin{enumerate}
\item \textbf{Formal Definition of Observer Types}:
\end{enumerate}
   
   \textbf{Type 1 Observer} ($O_1$): Chooses lower description levels
   - Selection weights: $w_0 >> w_1 >> w_2$
   - Focuses on "whether system reaches position", ignoring specific paths
   - State representation: $|\psi\rangle = \alpha_0|\hat{D}_0\rangle + \alpha_1|\hat{D}_1\rangle + ...$
   - Where $|\alpha_0|^2 >> |\alpha_1|^2 >> |\alpha_2|^2$
   
   \textbf{Type 2 Observer} ($O_2$): Chooses higher description levels
   - Selection weights: $w_2 >> w_1 >> w_0$
   - Focuses on "which specific path the system takes"
   - State representation: $|\psi\rangle = \beta_0|\hat{D}_0\rangle + \beta_1|\hat{D}_1\rangle + \beta_2|\hat{D}_2\rangle$
   - Where $|\beta_2|^2 >> |\beta_1|^2 >> |\beta_0|^2$

\begin{enumerate}
\item \textbf{Mathematical Derivation}:
\end{enumerate}
   
   Let the description state of the system reaching position $x$ through path $i$ be $|\text{path}_i, x\rangle$, with complex coefficient $A_i(x)$. Here $A_i(x)$ represents the weight of description state $|\text{path}_i, x\rangle$ in the total state $|\psi\rangle$.
   
   \textbf{Type 1 Observer} (Wave Properties):
   - Since paths are not distinguished, contributions from both paths must be added
   - Total amplitude: $A_{\text{total}}(x) = A_1(x) + A_2(x)$
   - Observed intensity: $I(x) = |A_1(x) + A_2(x)|^2$
   - Expanding: $I(x) = |A_1|^2 + |A_2|^2 + 2\text{Re}(A_1^*A_2)$
   - The third term $2\text{Re}(A_1^*A_2)$ produces interference pattern
   
   \textbf{Type 2 Observer} (Particle Properties):
   - Since paths are distinguished, each path's contribution is calculated independently
   - Path 1 intensity: $I_1(x) = |A_1(x)|^2$ (probability $p_1$)
   - Path 2 intensity: $I_2(x) = |A_2(x)|^2$ (probability $p_2$)
   - Total intensity: $I(x) = p_1|A_1|^2 + p_2|A_2|^2$
   - No cross terms, no interference

\begin{enumerate}
\item \textbf{Theoretical Explanation}:
\end{enumerate}
   
   "Wave-particle duality" in our theory is not a mysterious property of physical systems, but a necessary mathematical result of observers choosing different description levels in self-referential systems:
   - \textbf{Wave properties}: Observer chooses description level containing multi-path information
   - \textbf{Particle properties}: Observer chooses description level determining single path
   
   This explanation derives entirely from self-referential completeness and the entropy increase axiom, requiring no additional assumptions. $\square$

\section{Complete Derivation Chain from Self-Reference to Quantum}
\label{sec:ch05_quantum:complete-derivation-chain-from-self-reference-to-quantum}

\textbf{Theorem~\ref{thm:3.8} (Summary of Quantum Phenomena Necessity)}
\label{thm:3.8}
From "self-referentially complete systems necessarily increase entropy", the core features of quantum mechanics are logically necessary.

\textbf{Complete Derivation Chain}:
\begin{enumerate}
\item Self-referential completeness $\rightarrow$ Entropy increase (axiom)
\item Entropy increase $\rightarrow$ Dynamic process $\rightarrow$ Requires observer (Theorem~\ref{thm:3.1})
\item Observer self-reference $\rightarrow$ Multiple layers of description $\rightarrow$ Superposition (Theorems 3.2, 3.3)
\item Multiple descriptions $\rightarrow$ Weight coefficients $\rightarrow$ Quantum state amplitudes (Theorems 3.3, 3.5)
\item Observation selection $\rightarrow$ State collapse (Theorem~\ref{thm:3.4})
\item Observation type $\rightarrow$ Weight distribution $\rightarrow$ Wave-particle duality (Theorem~\ref{thm:3.7})
\end{enumerate}

\textbf{Theoretical Observations}:
\begin{itemize}
\item In our framework, quantum mechanical features appear as mathematical corollaries of self-referential systems
\item Observers are understood as structures emerging from entropy increase requirements
\item Wavefunction collapse is explained as a logical result of self-referential observation
\item Our proposed solution: treat observers as part of the system
\end{itemize}

\textbf{Theoretical Summary}: In our theoretical framework, quantum mechanical features appear as logical corollaries of self-referential systems. Observers are understood as intrinsic structures of the system's self-referential completeness.

\textbf{Deep Manifestation of Equivalence}:
Quantum phenomena perfectly demonstrate the unity of five-fold equivalence:
\begin{itemize}
\item \textbf{Observer $\Leftrightarrow$ Entropy increase}: Each measurement increases the system's information entropy
\item \textbf{Superposition $\Leftrightarrow$ Time undetermined}: Superposition represents multiple possibilities of time evolution
\item \textbf{Collapse $\Leftrightarrow$ Asymmetry}: Measurement creates before-after asymmetry
\item \textbf{Weights $\Leftrightarrow$ Information measure}: Quantum weight coefficients are essentially measures of information
\end{itemize}


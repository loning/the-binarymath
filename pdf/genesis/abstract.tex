
This paper constructs a theoretical framework based on a single axiom: \textbf{self-referentially complete systems necessarily increase in entropy}. This axiom explicitly defines core concepts including self-referential completeness, entropy, and information equivalence. From this clearly defined single axiom, we derive a series of theoretical results---including the form of information encoding ($\phi$-representation system), the observer mechanism in quantum mechanics, and mathematical structures formally similar to the Riemann Hypothesis. The framework's characteristic is: starting from an internally consistent single axiom and unfolding through rigorous logical derivation. We show that: (1) entropy increase requires optimal encoding, leading to the $\phi$-representation system; (2) self-reference requires self-observation, producing quantum collapse mechanisms; (3) the system's frequency balance leads to structures similar to the Riemann Hypothesis. This paper adopts a constructive view of truth, acknowledging the theory's constructed nature while emphasizing its internal consistency and explanatory power, rather than claiming to discover absolute truth.

\textbf{Keywords}: Single axiom, self-referential completeness, entropy increase principle, $\phi$-representation system, quantum observer, Riemann Hypothesis, constructive view of truth, information universe

\textbf{Mathematical Symbol Conventions}:
\begin{itemize}
\item $\mathbb{N}$: Set of natural numbers ${0, 1, 2, ...}$
\item $\log$: Natural logarithm with base $e$ (unless otherwise specified)
\item $\phi = \frac{1+\sqrt{5}}{2}$: Golden ratio
\item $|S|$: Cardinality of set $S$ (number of elements)
\end{itemize}

